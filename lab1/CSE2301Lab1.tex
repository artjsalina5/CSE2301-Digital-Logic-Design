\documentclass[options]{report}
\usepackage{graphicx}
\usepackage{amsmath}

\graphicspath{ {../images/} }



\newenvironment{Theory}
{
	\begin{center}
		\textbf{Theory\\}
	\end{center}
}

\newenvironment{Deliverables}
{
	\begin{center}
		\textbf{Deliverables\\}
	\end{center}
}

\newenvironment{Practice Questions}
{
	\begin{center}
		\textbf{Practice Questions\\}
	\end{center}
}

\author{Arturo Salinas-Aguayo}
\title{Lab 01: Logicworks}

\begin{document}
	\begin{titlepage}
		\begin{center}
			\vspace*{1cm}
			\LARGE
			\textbf{Lab 01: Logicworks Introduction\\}
			\vspace{8cm}
			\large
			\textbf{Arturo Salinas-Aguayo \\}
			\normalsize
			CSE 2301: Principles and Practice of Digital Logic Design\\
			Dr. Mohammad Khan Section 001\\
			\vspace{2cm}
			\includegraphics[scale=.1]{uconnlogo}
		\end{center}
	\end{titlepage}

	\begin{Theory}
		\begin{flushleft}
			\textbf{ORing Three Signals Using 2-Input OR Gates\\}
			To OR three signals \(C\), \(B\), and \(A\) together using only 2-input OR gates:
		\end{flushleft}
		\begin{enumerate}
			\item Connect signals B and A to the inputs of the first OR gate.
			\item The output of this OR gate is then connected to one input of the second OR gate.
			\item The remaining signal (C) is connected to the other input of the second OR gate.
		\end{enumerate}
		This arrangement ensures that the output of the circuit will be the logical OR of all three inputs. The truth table for this 3-way OR gate is given by:
		\vspace{2cm}
		\begin{center}
			\begin{tabular}{ c c c | c }
				C & B & A & F \\
				\hline
				0 & 0 & 0 & 0 \\
				0 & 0 & 1 & 1 \\
				0 & 1 & 0 & 1 \\
				0 & 1 & 1 & 1 \\
				1 & 0 & 0 & 1 \\
				1 & 0 & 1 & 1 \\
				1 & 1 & 0 & 1 \\
				1 & 1 & 1 & 1
			\end{tabular}
		\end{center}
	\end{Theory}
	\pagebreak

	\begin{Deliverables}
		\begin{flushleft}
		For Exercise 1, we were tasked with creating a circuit to acquaint ourselves with Logicworks 5. The circuit produced a truth table with the following inputs and outputs:\\
		\end{flushleft}
		\begin{center}
			\begin{tabular}{c c c |c c c| c}
				C & B & A & X & Y & Z & Hex \\
				\hline
				0 & 0 & 0 & 0 & 0 & 0 & 0x0 \\
				0 & 0 & 1 & 0 & 0 & 1 & 0x1 \\
				0 & 1 & 0 & 0 & 1 & 0 & 0x2 \\
				0 & 1 & 1 & 0 & 1 & 1 & 0x3 \\
				1 & 0 & 0 & 0 & 1 & 1 & 0x3 \\
				1 & 0 & 1 & 1 & 0 & 0 & 0x4 \\
				1 & 1 & 0 & 1 & 0 & 1 & 0x5 \\
				1 & 1 & 1 & 1 & 1 & 0 & 0x6
			\end{tabular}
		\end{center}
		\vspace{.5cm}
		\begin{flushleft}
			The hex output effectively converts a 3-bit binary input into 8 different possible combinations through the use of a Programmable Logic Array that weights the input values \(C\), \(B\), and \(A\) to 3, 2, and 1 respectively.  Each output line represents a "power of two" such that \(Z\) is \(2^0\), \(Y\) is \(2^1\), and \(X\) is \(2^2\). These are considered our "binary" output from which the Hex display device's inner logic renders the proper digit in our logicworks program. Hexadecimal is a base 16 radix system, and the conversion is straightforward as learned in class. Since \(C\) is worth a weight of 3, inputs \(100\) and \(001\) have the same XYZ and hex output.
		\end{flushleft}

		\textbf{Discussion}
		\begin{flushleft}
	This lab assignment had us gain familiarity with Logicworks 5 and our TA's. I have some experience using the Logisim Evolution open-source digital prototyping software, so it was a joy to learn a new tool, although a little bit disappointing that we will not be able to locally install this software on our computers. However, the lab was engaging as I gained familiarity with the class format and the gates' functions. The introduction to the circuit in the second example provided a new definition for me to learn, the 'minterm', which allowed us to create a sum-of-products form to design the circuit in Example 2.
		\end{flushleft}

	\end{Deliverables}
	\pagebreak
	\begin{Practice Questions}
		\textbf{Voltage Drop Across LED and Resistor in Series\\}
		- Given that the voltage drop across the LED is 4V:

		\begin{align*}
		V_{LED} &= 4V \\
		R &= 7000 \Omega \\
		V_{TOT} &= 18V \\
		V_{R} &= V_{TOT} - V_{R} = 18V - 4V = 14V \\
		V &= IR \Rightarrow I = \frac{V_{R}}{R} = \frac{14V}{7000\Omega} = 2mA
		\end{align*}



		\begin{flushleft}
			\textbf{TTL Logic Ranges\\}




		In TTL (Transistor-Transistor Logic): \\
		- Logic 0: A Voltage betweeen 0.8V and 2V is considered a logical `0`.\\
		- Logic 1: A voltage between 2V and 5V is considered a logical '1'.\\

		\vspace*{.5cm}

		This ensures the circuit can clearly distinguish between low (0) and high (1) signals. As a side effect, this makes TTL logic compatible with CMOS Logic as long as the 5V HIGH does not damage the receiver input.
		\end{flushleft}
\vspace{2cm}
	\end{Practice Questions}

\end{document}
