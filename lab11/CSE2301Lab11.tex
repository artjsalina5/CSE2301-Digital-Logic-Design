% TEX compiler = latexmk
% copyright arturo salinas-aguayo 2024
\documentclass[12pt]{article}

\usepackage{graphicx}
\usepackage{amsmath}
\usepackage{array}
\usepackage{amsfonts}
\usepackage{fancyhdr}
\usepackage{geometry}
\usepackage{circuitikz}
\usepackage{subfigure}
\usepackage{caption}
\usepackage{karnaugh-map}
\usepackage{bm}
\usepackage[table]{xcolor}
\usepackage{float}
\usepackage{subcaption}

\geometry{letterpaper, margin=1in}
\graphicspath{ {../images/} }

% Header and Footer
\pagestyle{fancy}
\fancyhf{}
\fancyhead[L]{CSE 2301 - Lab 11: Programmable Logic Arrays}
\fancyhead[R]{\thepage}
\setlength{\headheight}{15pt}

\author{Arturo Salinas-Aguayo}
\title{Lab 11: Programmable Logic Arrays}
% theorem set
\newtheorem{example}{Example}
% Example block environment
\newenvironment{examp}
{
	\vspace{.5cm}
	\hrule
\begin{example}\upshape}
	{\hrule
		\vspace{0.5cm}
\end{example}}

\begin{document}
\newcommand{\closure}[2][3]{%
	{}\mkern#1mu\overline{\mkern-#1mu#2}}
\newcommand\ncoverline[1]{\mkern1mu\overline{\mkern-1mu#1\mkern-1mu}\mkern1mu}
% Title Page
\begin{titlepage}
	\centering
	\vspace*{3cm}
	\huge\textbf{Lab 11: Programmable Logic Arrays}\\
	\vspace{5cm}
	\Large\textbf{Arturo Salinas-Aguayo}\\
	\normalsize
	CSE 2301: Principles and Practice of Digital Logic Design\\
	Dr. Mohammad Khan, Section 003L-1248\\
	Electrical and Computer Engineering Department
	\vfill
	\includegraphics[scale=0.1]{uconnlogo}\\
	College of Engineering, University of Connecticut\\
	\scriptsize{Coded in \LaTeX}
	\vspace*{1cm}
\end{titlepage}
\section*{Theory}
\subsection*{Karnaugh-Map Review}
\textbf{F Output}
\begin{center}
\begin{karnaugh-map}(label=corner)[4][4][2][$B$][$A$][$C_0$][$C_1$][$C_2$]
\minterms{0,1,2,3,5,6,7,8,9,10,13,14,16,19,23,24}
\implicant{1}{9}
\implicant{2}{10}
\implicant{3}{7}[0,1]
\implicantedge{0}{0}{8}{8}[0,1]
\autoterms[0]
\end{karnaugh-map}
\end{center}

\[
	F = \closure{C_0}\closure{A}\closure{B} + \closure{C_1}AB +
	\closure{C_2}\closure{A}B + \closure{C_2}A\closure{B}
\]
\begin{table}[h!]
	\centering
	\newcommand{\currstatecolor}{gray!30}
	\begin{tabular}{|c|c|c|c|c
		|>{\columncolor{\currstatecolor}}c
		|>{\columncolor{\currstatecolor}}c
		|>{\columncolor{\currstatecolor}}c
		|>{\columncolor{\currstatecolor}}c
		|l|}
		\hline
		\textbf{A} & \textbf{B} & \textbf{C} & \textbf{D} & \textbf{E} & \textbf{F} & \textbf{G} & \textbf{H} & \textbf{I} & \textbf{Error Position} \\ \hline
		0          & 0          & 0          & 0          & 0          & 0          & 0          & 0          & 0          & None (No Error)         \\ \hline
		1          & 0          & 0          & 0          & 0          & 0          & 1          & 0          & 1          & Parity Bit 5            \\ \hline
		1          & 1          & 0          & 0          & 0          & 0          & 1          & 0          & 0          & Data Bit 4              \\ \hline
		1          & 0          & 1          & 0          & 0          & 0          & 0          & 1          & 1          & Data Bit 3              \\ \hline
		1          & 0          & 0          & 1          & 0          & 0          & 0          & 1          & 0          & Data Bit 2              \\ \hline
		1          & 0          & 0          & 0          & 1          & 0          & 0          & 0          & 1          & Data Bit 1              \\ \hline
		0          & 1          & 0          & 0          & 0          & 1          & 0          & 0          & 1          & Parity Bit 9            \\ \hline
		0          & 1          & 1          & 0          & 0          & 1          & 0          & 0          & 0          & Data Bit 8              \\ \hline
		0          & 1          & 0          & 1          & 0          & 0          & 1          & 1          & 1          & Data Bit 7              \\ \hline
		0          & 1          & 0          & 0          & 1          & 0          & 1          & 1          & 0          & Data Bit 6              \\ \hline
		0          & 0          & 1          & 0          & 0          & 1          & 1          & 0          & 0          & Parity Bit 12           \\ \hline
		0          & 0          & 1          & 1          & 0          & 1          & 0          & 1          & 1          & Data Bit 11             \\ \hline
		0          & 0          & 1          & 0          & 1          & 1          & 0          & 1          & 0          & Data Bit 10             \\ \hline
		0          & 0          & 0          & 1          & 0          & 1          & 1          & 1          & 0          & Parity Bit 14           \\ \hline
		0          & 0          & 0          & 1          & 1          & 1          & 1          & 0          & 1          & Data Bit 13             \\ \hline
		0          & 0          & 0          & 0          & 1          & 1          & 1          & 1          & 1          & Parity Bit 15           \\ \hline
	\end{tabular}
	\caption{Truth Table for Error Detection (5 Inputs, 4 Outputs)}
\end{table}

\end{document}
% vim: set tw=80 ts=2 sts=2 sw=2 noai noet:
